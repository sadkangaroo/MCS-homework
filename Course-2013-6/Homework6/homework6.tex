\documentclass[a4paper, 12pt]{mcshw}
\begin{document}
\Letsmaketitle{6}
\begin{enumerate}
    \item Modify the proof that every monotone peroperty has a threshold for $G(n, p)$ to apply to the 3-CNF satisfiability problem
        \begin{solution}
            Let's call the 3-CNF model $B(n, m)$.

            We define an increasing property of $B(n, m)$ as follows: 
            \begin{quote}
                $Q$ is an increasing property of $B$ if when a 3-CNF $B$ has the property any 3-CNF obtained by adding clauses to $B$ must also have the property.
            \end{quote}
            First we will proof the following lemma: 
            \begin{quote}
                If $Q$ is an increasing property and $0 \leq m_1 \leq m_2 \leq 1$, then the probability that $B(n, m1)$ has property Q is less than or equal to the probability that $B(n, m2)$ has property Q.
            \end{quote}
            Notice we could generate $B(n, m_2)$ in this way: First we generate $B(n, m_1)$, then we generate a set $B(n, m_2 - m_1)$ and take the intersection of them to get $B(n, m_2)$. If $B(n, m_1)$ has the property $Q$, $B(n, m_2)$ must has it too. 

            Next we will prove
            \begin{quote}
                Every increasing property of $B(n, m)$ has a threshold at $m(n)$, where for each $n$, $m(n)$ is the minimum real number $a$ for which the probability that $B(n, a)$ has the property $Q$ is $\frac{1}{2}$.
            \end{quote}
            Suppose $m_0(n)$ is any function such that
            $$\lim_{n\to\infty}\frac{m_0(n)}{m(n)} = 0$$
            We will show that almost surely $B(n, m_0)$ does not have property Q. Suppose this is false. Then, the probability that $B(n, m_0)$ has the property Q does not converge to zero. By the definition of limit, there must be a positive real number $\epsilon$ for which the probability that $B(n, m_0)$ has property Q is at least $\epsilon$ on an infinite set $I$ of $n$. 
            
            Here we define the m-fold replication of $B$ be the intersection of $m$ copies of $B(n, m)$.

            Let $k = \lceil (1/\epsilon) \rceil$. Let 3-CNF $H$ be the m-fold replication of $B(n, m_0)$. Since from the m-fold method we have
            $$Prob\bigl(B(n, km)\text{ does not have Q}\bigr) \leq {\Bigl(Prob\bigl(B(n, m)\text{ does not have Q}\bigr)\Bigr)}^k$$
            Then the probability that H does not have $Q$ is at most ${(1 - \epsilon)}^m \leq e^{-1} \leq 1/2$ for all $n \in I$. So for these $n$, since $m(n)$ is the minimum real number $a$ for which the probability that $B(n, a)$ has property $Q$ is $1/2$, $km_0 \geq m(n)$. This implies that $\frac{m_0(n)}{m(n)}$ is at leat $1/k$ infinitely often contradicting the hypothesis that $\lim\limits_{n\to\infty}\frac{m_0(n)}{m(n)} = 0$. A symmetric argument shows that for any $m_1(n)$ such that $\frac{m(n)}{m_1(n)}\to 0$, $B(n,m_1)$ almost surely has property Q.
        \end{solution}
    \item Verify that the sum of $r$ rank-one matrices $\sum_{i = 1}^r\sigma_iu_iv_i^T$ can be written as $UDV^T$, where the $u_i$ are the columns of $U$ and $v_i$ are the columns of $V$. To do this, first verify that for any two matrices $P$ and $Q$, we have
        $$PQ^T = \sum_ip_iq_i^T$$
        where $p_i$ is the $i^{th}$ column of $P$ and $q_i$ is the $i^{th}$ column of $Q$.
        \begin{proof}
            First we will prove
            $$PQ^T = \sum_ip_iq_i^T$$

            Let $A = PQ^T$

            Then 
            $$a_{ij} = \sum_{k = 1}^rp_{ik}q_{jk}$$
            Let $B = \sum\limits_ip_iq_i^T$

            Then 
            \begin{align*}
                b_{ij} &= \sum_{k = 1}^r\{p_kq^T_k\}_{ij}\\
                &= \sum_{k = 1}^rp_{ik}q_{jk}\\
                &= a_{ij}
            \end{align*}
            Do some tiny modification, we can complete our proof.

            Let $A = UDV^T$

            Then 
            $$a_{ij} = \sum_{k = 1}^ru_{ik}d_{kk}v_{jk}$$
            Let $B = \sum\limits_i\sigma_iu_iv_i^T$

            Then 
            \begin{align*}
                b_{ij} &= \sum_{k = 1}^r\{\sigma_ku_kv_k^T\}_{ij}\\
                &= \sum_{k = 1}^ru_{ik}d_{kk}v_{jk}\\
                &= a_{ij}
            \end{align*}
            Thus \ $UDV^T = \sum\limits_{i = 1}^r\sigma_iu_iv_i^T$
        \end{proof}
\end{enumerate}
\end{document}
